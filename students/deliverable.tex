\documentclass{article}

\usepackage{xcolor}
\usepackage{tikz}
\usepackage{graphicx}
\usepackage{float}

\begin{document}

\section{Casos de uso}

Dadas las condiciones del problema original se desarrollo el siguiente
diagrama de casos de uso:

\begin{figure}[H]
    \centering
    \scalebox{.75}{
        \input{uml/tex/courses_002.tex}
    }
\end{figure}

Podemos ver que habrá entonces 3 acciones posibles:
\begin{itemize}
    \item Un alumno puede anotarse en un curso.
    \item Un profesor puede evaluar a un alumno.
    \item Un tutor puede conseguir la nota media de un alumno.
\end{itemize}

\pagebreak
De todas las acciones posibles, que un alumno se anote a un curso es la más
elaborada, por lo tanto, podemos utilizar un diagrama de acitividad para
entender mejor el proceso:
\begin{figure}[H]
    \centering
    \scalebox{.75}{
        \input{uml/tex/courses_001.tex}
    }
\end{figure}

\pagebreak
Finalmente, con el diagrama de casos de uso y el diagrama de acitividad, se
puede plantear un diagrama de clases como el siguiente:
\begin{figure}[H]
    \centering
    \scalebox{.5}{
        \input{uml/tex/courses.tex}
    }
\end{figure}

\end{document}
