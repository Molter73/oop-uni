\documentclass[a4paper,notitlepage]{article}

\usepackage{xcolor}
\usepackage{tikz}
\usepackage{graphicx}
\usepackage{float}
\usepackage{fullpage}

\begin{document}

Esta versión de Mastermind utilizará juegos de 12 intentos con un total de
5 clavijas por intento. La implementación se puede realizar completamente por
composición, el elemento más pequeño será una clavija (Peg), la cual poseerá
solamente un campo color. En base a la clavija, podemos crear una clase
fila (Row) que poseerá un array de 5 clavijas. Podemos así definir un tablero
(Board) como una agrupación de filas:

\begin{itemize}
    \item Un array de 12 filas contendrán los intentos que vaya utilizando el usuario.
    \item Un segundo array de 12 filas tendrá las pistas para cada intento realizado.
    \item Una última fila tendrá la solución al juego y no será accesible hasta que el juego termine.
\end{itemize}

Vemos entonces que el tablero tendrá un total de 25 filas. Con el propósito de
poder jugar el juego, se implementará una clase Game, la cual preguntará al
usuario cuántos juegos desea jugar e iterará esa cantidad de veces en la
siguiente secuencia:

\begin{enumerate}
    \item Instanciar un tablero nuevo (el tablero generará el código a adivinar al instanciarse).
    \item Se pedirá al usuario que ingrese la fila de clavijas que desea jugar.
    \item Se pasará la fila al tablero mediante el método tryAnswer() y este devolverá la fila de pistas.
    \item Si se adivinó el código o no hay más intentos, el juego termina y se actualiza la calificación, caso contrario vuelve al punto 2.
\end{enumerate}

Una vez terminados todos los juegos que el usuario quiere jugar, se imprimirá
la calificación final por pantalla.

\pagebreak
Podemos entonces plantear un diagrama de clases como el siguiente:

\begin{figure}[H]
    \centering
    \scalebox{.6}{
        \input{build/uml.tex}
    }
\end{figure}

\end{document}
