\documentclass{article}

\usepackage{xcolor}
\usepackage{tikz}
\usepackage{graphicx}
\usepackage{float}

\author{Mauro Ezequiel Moltrasio}

\title{Diseño de un ATM}

\begin{document}

\maketitle

En base al enunciado del problema, se pueden identificar 5 casos de uso
representados en el siguiente diagrama:

\begin{figure}[H]
    \centering
    \scalebox{.5}{
        \input{build/uml.tex}
    }
\end{figure}

Se procede entonces a definir dos clases principales:
\begin{itemize}
    \item Account: mantendrá el saldo de una cuenta y proveerá métodos para
        interactuar con el mismo.
    \item ATM: Funcionará como interfaz con el usuario, traduciendo el input
        del usuario en mensajes para la cuenta e imprimiendo el resultado de
        operaciones para que el usuario tenga feedback.
\end{itemize}

Adicionalmente, para poder manejar errores se definirán algunas excepciones
personalizadas para que la cuenta pueda comunicar errores al ATM.

Finalmente, las opciones del menú principal se definirán como un enum,
garantizando así que todos los casos están manejados.

El diagrama de clases final queda como sigue:

\begin{figure}[H]
    \centering
    \scalebox{.5}{
        \input{build/uml_001.tex}
    }
\end{figure}

\end{document}
