\documentclass{article}

\usepackage{xcolor}
\usepackage{tikz}
\usepackage{graphicx}
\usepackage{float}

\begin{document}

Basado en la descripción del ejercicio y el código provisto como ejemplo, se
decidió crear 4 clases:
\begin{itemize}
    \item Flight: Contendrá la información básica de un vuelo.
    \item Client: Tendrá la información referida a un cliente.
    \item Booking: Vinculará un cliente con un vuelo para generar una reserva.
    \item App: Contendrá la lógica principal de la aplicación.
\end{itemize}

La necesidad de la clase Flight resulta evidente del enunciado y viendo que
la mayor parte de variables existentes en la aplicación se reliacionan
directamente con ella.

La clase Client decidió crearse pensando que a futuro podría necesitarse
guardar más información de un cliente que sólo su número.

Se eligió representar las reservas como una clase a parte que
contiene el vuelo y el cliente que ha realizado la reserva. Se podría haber
optado por tener las reservas vinculadas al cliente, pero si a
futuro se quisiera implementar búsquedas del tipo reservas en el vuelo X,
deberíamos recorrer todos los clientes para encontrar las reservas de un vuelo.
Una situación similar se daría si las reservas se vinculan a un vuelo y
queremos implementar una búsqueda de reservas para el cliente Y, deberíamos
recorrer todos los vuelos para encontrar las reservas del cliente. Tenerlas
como entidad separada pero con referencias a cliente y vuelo permite
implementar ambas búsquedas recorriendo un sólo contenedor.

Finalmente, la clase App crea los contenedores necesarios y les une con la
lógica de negocio que permite crear reservas para los vuelos.

\begin{figure}[H]
    \centering
    \scalebox{.5}{
        \input{build/uml.tex}
    }
\end{figure}

\end{document}
